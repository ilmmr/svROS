% book example for classicthesis.sty
\documentclass[
  % Replace twoside with oneside if you are printing your thesis on a single side
  % of the paper, or for viewing on screen.
  oneside,
  %twoside,
  11pt, a4paper,
  footinclude=true,
  headinclude=true,
  cleardoublepage=empty
]{scrbook}

\setlength{\marginparwidth}{2cm}

\usepackage{indentfirst}
\usepackage{dissertation}
\usepackage{float}
%---
\usepackage[T1]{fontenc}
\usepackage{textcomp}
%---
\usepackage[titles]{tocloft}
%% Aesthetic spacing redefines that look nicer to me than the defaults.
\setlength{\cftbeforechapskip}{2ex}
\setlength{\cftbeforesecskip}{0.5ex}
%% Use Helvetica-Narrow Bold for Chapter entries
\renewcommand{\cftpartfont}{%
  \fontsize{12}{14}\usefont{OT1}{phv}{bc}{n}\selectfont
}
\renewcommand{\cftchapfont}{%
  \fontsize{11}{13}\usefont{OT1}{phv}{bc}{n}\selectfont
}
\renewcommand{\cftsecfont}{%
  \fontsize{10}{11}\usefont{OT1}{phv}{}{n}\selectfont
}
\renewcommand{\cftsubsecfont}{%
  \fontsize{9}{10}\usefont{OT1}{phv}{}{n}\selectfont
}
\renewcommand{\cftfigfont}{%
  \fontsize{9}{10}\usefont{OT1}{phv}{}{n}\selectfont
}
\renewcommand{\cfttabfont}{%
  \fontsize{9}{10}\usefont{OT1}{phv}{}{n}\selectfont
}
%---
%usepackage[scaled=.92]{helvet}
\usepackage[all]{xy}
\usepackage{circuitikz}

\titleclass{\subsubsubsection}{straight}[\subsection]


% Title

\titleA{Formalizing ROS2 security configuration with Alloy}

% Author

\author{Luís Mário Macedo Ribeiro}

% Supervisor(es)

\supervisor{Manuel Alcino Pereira da Cunha}

\cosupervisorA{André Filipe Faria dos Santos}

% Date

\date{\myear} % change to text if date is not today

%\makeglossaries %  either use this ...

\makeindex	% ... or this

\begin{document}\fontfamily{phv}\fontseries{mc}\selectfont

% Add acronym definitions

% Cover page ---------------------------------------------
%	\thispagestyle{empty}
	%!TEX root = dissertation.tex

\makeatletter

% UM_ENg Logo
\def\UMEng#1#2{\begin{tikzpicture}[
	% bars styling,
	logone/.style={rectangle,fill=white,rounded corners=0.08cm,minimum width=0.16cm,inner sep=0pt},
	bigone/.style={minimum height=0.74cm},
	smaone/.style={minimum height=0.48cm},
	engone/.style={minimum height=0.86cm},
	pos1/.style={xshift=1.3cm,yshift=1.3cm},
	pos2/.style={xshift=3.9cm,yshift=1.3cm}]
	
% Uminho logo
	\fill[fill=#1] (0,0) -- (2.6,0) -- (2.6,2.6) -- (0,2.6) -- cycle;
	\foreach \i in {1,...,3}{
		\node at (\i*120+30:0.45)[logone,bigone,pos1,rotate=\i*120-60]{};
		\node at (\i*120+90:0.60)[logone,smaone,pos1,rotate=\i*120]{};
	}

% EngUminho logo
	\fill[fill=#2] (2.6,0) -- (5.2,0) -- (5.2,2.6) -- (2.6,2.6) -- cycle;
	\foreach \i in {1,...,5}
		\node at (\i*72-90:0.74)[engone,logone,pos2,rotate=\i*72-90]{};
\end{tikzpicture}}

\def\yyy#1{\fontfamily{phv}\fontseries{mc}\selectfont {\ifnum\hide=1\relax\else#1\fi}}
\def\xxx#1{\fontfamily{phv}\fontseries{mc}\selectfont #1}
\def\zzz#1{\fontfamily{phv}\fontseries{mc}\fontseries{b}\selectfont #1}
\def\kkk#1{\fontfamily{phv}\fontseries{mc}\fontseries{b}\selectfont {\ifnum\hide=1\relax\else#1\fi}}

\long\def\coverEtc{
%Logo
~\vskip-4.1cm\rule{4cm}{0pt}\begin{tabular}{l}
\UMEng\umc{eng}
\\\zzz{Universidade do Minho}\rule{0pt}{1cm}
\\\xxx{}{Escola de Engenharia}
\\\xxx{Departamento de  Informática}
\\\rule{0pt}{4cm}
\\\xxx{{\Large\@author}}
\\\rule{0pt}{1em}
\\\zzz{\Large\@titleA}
\\\zzz{\Large\@titleB}
\\\zzz{\Large\@titleC}
\\\rule{0pt}{5cm}
\\\yyy{\large Master dissertation}
\\\yyy{\large Integrated Master’s in Informatics Engineering}
\\\rule{0pt}{6mm}
\\\yyy{\large Dissertation supervised by}
\\\kkk{\@supervisor}\rule{0pt}{4mm}
\\\kkk{\@cosupervisor}
\\\rule{0pt}{4.2cm}
\\\xxx{{\small\@date}}
\end{tabular}
}


\begin{frontcover}
\gdef\umc{um}\gdef\hide{1}
\thispagestyle{empty} \pagecolor{white} \textcolor{black} \coverEtc
\end{frontcover}

\begin{titlepage}
\gdef\umc{um}
\gdef\hide{0}
\thispagestyle{empty} \pagecolor{white}\textcolor{grey} \coverEtc
\end{titlepage}

\makeatother


%rm
	\cleardoublepage
%---------------------------------------------------------
	\pagenumbering{alph}
	\setcounter{page}{0}
%---------------------------------------------------------
% Add acknowledgements

\chapter*{Copyright and Terms of Use for Third Party Work}

\noindent This dissertation reports on academic work that can be used by third parties as long as the internationally accepted standards and good practices are respected concerning copyright and related rights.
\vskip 1em
\noindent This work can thereafter be used under the terms established in the license below.
\vskip 1em
\noindent Readers needing authorization conditions not provided for in the indicated licensing should contact the author through the RepositóriUM of the University of Minho.

\section*{License granted to users of this work:}

\CCBY % or replace by one in***************** the list below, cf https://alunos.uminho.pt/PT/estudantes/Formataes/3_Despacho_RT-31_2019_Anexo%203-Informa%c3%a7%c3%a3o-Direitor%20de%20Autor.docx
%---------
%\CBYNCND
%\CCBYNCSA
%\CCBYNC
%\CCBYND
%\CCBYSA


%---------

\chapter*{Acknowledgements}
% Write your acknowledgements here. Do not forget to mention the projects and grants that you have benefited from while doing your research, if any. Ask your supervisor about the specific textual format to use. (Funding agencies are quite strict about this.)

	\cleardoublepage

%---------

\chapter*{Statement of Integrity}

\noindent I hereby declare having conducted this academic work with integrity.
\vskip 1em\noindent
I confirm that I have not used plagiarism or any form of undue use of information or falsification of results along the process leading to its elaboration.
\vskip 1em\noindent
I further declare that I have fully acknowledged the Code of Ethical Conduct of the University of Minho.

%---------

% Uncomment as wished

% Add abstracts (en,pt) -----------------------------------------------------------
\chapter*{Abstract}
	
	Automation developments are becoming essential to industrial restructuring, as it brings more efficient and accurate processes with less associated cost. Consequently, robots are rapidly being deployed in a wide range of scenarios, especially where security is demanded. In such cases, it is critical to employ appropriate procedures to verify both the system's quality and its security.

	Following the current growth of cyber-physical system, as well as their usage into various technology domains, the development of software applications is demanding due to the complexity behind the integration of needed services, beyond those provided by the operating system. Hereupon, middleware software to abstract systems hardware are constantly evolving, while offering services that support application development and delivery.

	One of the most popular open-source software platforms for building robotic systems is the Robot Operating System (ROS)\cite{1} middleware, where highly configurable robots are usually built by composing third-party modules. A major factor behind its popularity and widespread adoption is its flexibility and interoperability. One drawback of this flexibility, however, lies in the increased security risks that ROS applications face. Alongisde security risks, the arising of performance and scalability issues related to the ROS middleware specification, forced the creation of a new version of ROS.

	Robot Operating System 2 (ROS2), which continues to provide a simple, uniform message passing interface to allow components to communicate with each other, now implemented using the Data Distribution Service (DDS)\cite{3} communication protocol, where security guarantees are ensured by DDS-Security specification. Using DDS-Security, it is possible to configure ROS2 to run with security guarantees using the SROS2 toolset \cite{ros-dds-integration}. However, improper configuration can still lead security problems.
	
	This thesis proposes a technique, based on the software verification perspective, to automatically verify system-wide properties related to the security configuration of ROS2-based applications. The intended purpose is to model the ROS architecture, as well as the network communication behaviour, in Alloy\cite{alloy-DJ}, a formal specification language and analysis tool supported by a model-finder over which, system-wide properties are subsequently model-checked.

\paragraph{Keywords} Robotics, ROS, ROS2, DDS, SROS2, Security, Software Verification, Alloy
	\cleardoublepage

\chapter*{Resumo}

	A constante implementação da ideia de automização de processos tem motivado a reestruturação nos mais diversos setores industriais, com o obejtivo de aumentar a eficiência e precisão nos processos integrados, consequentemente, reduzindo os custos associados. Consequentemente, impulsionam a integração robótica nos mais amplos domínios tecnológicos, especialmente em domínios onde a segurança é exigida. Nestes cassos, é fundamental adotar técnicas apropriadas de forma a verificar tanto a qualidade do sistema, como a segurança do mesmo.

	Como resultado do atual crescimento de sistemas sistemas ciber-físicos, nomeadamente sistemas robóticos, bem como sua utilização em vários domínios tecnológicos, o desenvolvimento de aplicações é exigente devido à complexidade da integração dos serviços necessários, tipicamente não fornecidos pelo sistema operativo. De forma a acompanhar o aumento na complexidade destes sistemas, middlewares que permitem abstrair hardware têm sido adoptados, oferecendo serviços que oferecem suporte ao desenvolvimento de aplicações robóticas.

	Uma das plataformas considerada como \textit{standard} no que toca ao desenvolvimento sistemas robóticos é o middleware Robot Operating System (ROS)\cite{1}, onde robôs altamente configuráveis são construídos atráves da composição modular de \textit{software} externo, oferencedo características como flexibilidade e interoperabilidade aos sistemas integrados. No entanto, a flexibilidade resulta num aumento de vulnerabilidades de segurança, pondo em causa a integridade das aplicações. Além da falta de segurança apresentada, existem também problemas de desempenho e escalabilidade relacionados com a especificação do middleware. Assim, era necessário uma mudança na estruturação do ROS, resultando na criação do Robot Operating System 2 (ROS2).

	A nova versão do ROS, o Robot Operating System 2 (ROS2) implementa um protocolo de comunicação de nome Data Distribution Service (DDS)\cite{3}, onde para além de garantir serviços de comunicação, fornece diversas especificações, sendo uma delas a especificação DDS-Security, que através de uma metodologia de \textit{plugins}, oferece diferentes métodos de adoção de segurança. Através do uso desta especificação, juntamente com o uso do SROS2 \textit{toolset} \cite{ros-dds-integration}, é possível configurar o ROS2 de forma a adotar estas medidas de segurança.

	Esta tese propõe uma técnica para a verificação automática de \textit{system-wide properties} em aplicações ROS. Esta técnica apresentada baseia-se na formalização estrutural de arquiteturas ROS em Alloy\cite{alloy-DJ}, com o obejtivo de modelar o comportamento associado à comunicação dentro do sistema, tendo em consideração configurações associadas às propriedades de segurança. 
	
	
\paragraph{Palavras-chave} Robótica, ROS, ROS2, DDS, SROS2, Segurança, Verificação de Software, Alloy


	\cleardoublepage

	\pagenumbering{roman}
	\setcounter{page}{3}
	%pagestyle{fancy}   % -------- removed
	%rm

	% Document
	\cleardoublepage
    \phantomsection
	\tableofcontents

	\cleardoublepage
	\listoffigures

	%\cleardoublepage
	%\listoftables

	%\cleardoublepage
	%\lstlistoflistings

	% Add list of acronyms
	\cleardoublepage
	\pagenumbering{arabic}
	\setcounter{page}{3}

\part{Introductory material}

\chapter{Introduction}\label{c:intro}

The concept behind automation development is being incorporated into the industrial world, through the use of flexible tools to assist in the most various scenarios, as it brings efficiency and accuracy to the industry's processes. Robotics is already the key driver of competitiveness and flexibility in large scale manufacturing industries, as it is significantly reliant on a variety of technologies. Due to the continuous growth of technology in these different domains, robots can be used in a wide range of applications \cite{mohamed2008middleware}, since their usage brings increased productivity, safety and more manufacturing production work back to developed countries \cite{robots-importance, craig2005introduction}.

Despite the advances in technology, dealing with hardware-level applications becomes highly impractical as the system's complexity increases. Thereupon, developing and writing software code for robot applications is demanding, where multiple aspects must be properly considered \cite{intro-ros}. 

Since robots became to be integrated into distributed systems through separated components, connecting different hardware and software modules raises interoperability and communication issues. To solve this issue, modular architectures, based on message-passing communication patterns, are continually emerging as the architecture's middleware layer. Their primary focus is to offer services to the application layer, consequently easing the development cost, while providing interoperability and communication facilities \cite{mohamed2008middleware, maruyama2016exploring}. The requirement for a middleware layer that meets different robot's specification is a novel approach to enable the creation of robot applications over robotic systems, while supporting features such as robustness and modularity. 

% Formerly, robots were designed to complete certain single tasks, so they tended to build as one unit. However, robots are now composed by components concerned by a shared distributed network, promoting the ideia of connecting different hardware and software modules that provide control over these components. The integrated robot's modules coop together to complete a shared purpose. Since these software modules can have distinguished specification, the integration between them raises interoperability and communication issues, so the complexity behind the robots application development must be addressed by a middle layer, the \textbf{middleware}. The main idea behind implementing a middleware into a distributed system, is to ease the development cost, while provide interoperability and communication facilitation.\cite{mohamed2008middleware} The requirement for a middleware layer that meets different robot's specfication is a novel approach to enable the creation of robot applications over robotic systems, while supporting features such as robustness and modularity. 

% Thereupon, developing and writing software code for robot applications is demanding, where multiple aspects must be considered properly, as the complexity of robotics is continuously growing. Robots tend to be designed for a particular purpose, resulting in a wide variety of hardware combined, consequently making already written software code extremely difficult to understand and implement, so the complexity behind is demanding. Typically, code perception and reasoning are way to complex for any single programmer, when abstraction strategies are not taking into account.\cite{intro-ros} % Even though, many robotics researchers have previously created frameworks to deal with these problems, simplifying complexity issues by providing rapid prototyping, they tend to not scale to feature a wide community of robotics programmers. ROS aims to solve this concern, by providing a modular package-based framework, designed to be built upon by robot software developers. Their software can then be utilized by a variety of platforms and applications.\cite{intro-ros}

The Robot Operating System was created by a collaborative open-source community to contribute in the advancement of robots, with the aim of helping build robot applications easily \cite{diluoffo2018robot}. It enables locomotion, manipulation, navigation, and recognition tasks over complemented software libraries and tools. Concerning the wide range of robotics hardware and software, ROS was designed to be flexible, enabling interpolation with potential added components. However, performance and scalability issues arise due to its middleware specification \cite{intro-ros}. Additionally, real-time constraints such as fault-tolerance, deadlines, or process synchronization were not supported by ROS, making it unsuitable for safety-critical and real-time systems \cite{kim2018security}.

Besides having no middleware support for distributed real-time systems, security was not prioritized by ROS, which started to be demanding for deployed systems. An increasing number of real-time applications, for instance robotic systems, requires security insurance for protecting real-time sensitive data against unauthorized access \cite{lin2009static}.
% Previously, robotic systems were protected using a static environment, usually closed networks. 
% The need for robotics evolution demanded a change of approach, where systems must be acessible and extendable from the public network, allowing flexibility within the network, at the cost of security.

This lead to the creation of Robot Operating System 2, developed using the Data Distribution Service (DDS) \cite{3} specification protocol as its middleware, leveraging for its messaging architecture. Issues concerning system integration and scalability are mitigated by DDS various implementations, due to the several transport configurations provided, making it suitable for real-time distributed systems. DDS also provides a security specification, called DDS-Security. ROS2 makes use of this specification, providing security guarantees to the deployed robotic systems \cite{8442103}.

Due to the widespread usage of robotic systems, software verification, through the use of formal methods, are necessary to prevent potentially catastrophic consequences, mainly related to security matters, as safety guards are gradually implemented into the software domain \cite{wu2017safety}. Within this context, Alloy \cite{alloy-DJ, lwspecification} framework enables the behavioural representation of systems with rich configurations, due to the combination of both relational and linear temporal logic (LTL) provided by its specification language, consequently supporting model-checking techniques. Model-checking techniques enable far better levels of coverage and, as a result, more reliability than traditional testing, where the system is abstracted as a conventional model, that is automatically checked over performing property verification on finite-state machines \cite{beyer2017software}.

% Thus, a suitable approach towards the analysis and verification of safety-critical properties on real systems, should start through the description of a pipeline, that considers the real system extraction to an abstract model and the analysis of some concrete pre-defined properties, and lastly, the verification results proper inspection. Since most users of low-level programming languages have no familiarity with Formal Methods, aiming to achieve high impact in the improvement of software quality, formal techniques may be integrated through lightweight formal approaches, where completeness is sacrificed in favor of potential automation. One of the major limitations to this are usually associated with the real domain of action by the systems under analysis. Usually, for capturing relevant elements to the analysis, it’s required to have high-knowledge of the execution environment and systems reversing engineering. Such processes have weak potential for automation.

The proposal of this dissertation is to develop a novel technique to automatically verify system-wide safety properties using Alloy framework, confining a ROS2 system into an abstract model, in order to obtain a prototype tool that can be used by developers to easily detect security configuration issues on their respective robotic application.



% \section{Problem Statement}\label{s:problem}

% As reliance on robotic systems increases,  
% Robotic systems, so called \textbf{Cyber-physical} systems, besides concerning about its realiability as a trusted system, must prioritize 
% concerns such as time efficiency and distribution are considered as priority. % Systems that rely on distribution, referred as a distributed system, supports the idea of no centralization of the systems components, by dispersing them across the systems network and handling communication by message-passing. 
% System distribution relates to the productivity matters, making robotic systems more flexible, by allowing robot compoisiton into the same network. Systems that relie on time effiency meets real-time constraints, called deadlines, mainly concerns about safety, since robots can directly affect human lifes. However, robotic systems need to be responsive, to expect maximum productivity, especially when its being considered a distributed network that rely on different robotic components. 

% Formerly, robots were designed to complete certain single tasks, so they tended to build as one unit. However, robots are now composed by components concerned by a shared distributed network, promoting the ideia of connecting different hardware and software modules that provide control over these components. The integrated robot's modules coop together to complete a shared purpose. Since these software modules can have distinguished specification, the integration between them raises interoperability and communication issues, so the complexity behind the robots application development must be addressed by a middle layer, the \textbf{middleware}. The main idea behind implementing a middleware into a distributed system, is to ease the development cost, while provide interoperability and communication facilitation.\cite{mohamed2008middleware} The requirement for a middleware layer that meets different robot's specfication is a novel approach to enable the creation of robot applications over robotic systems, while supporting features such as robustness and modularity. 

%In real-time distributed systems, middleware for robotics development must meet stringent requirements.\cite{maruyama2016exploring} The former version of ROS, was previously considered as an open-source middleware, as they provided their own middleware implementation. However, performance and scalability issues arised due to its middleware specification.\cite{intro-ros} Additionally, real-time constraints such as fault-tolerance, deadlines, or process synchronization were not supported by ROS, making it unsuitable for safety-critical and real-time systems.\cite{kim2018security} 

% Besides having no middleware support for distributed real-time systems, security features were also not featured by ROS, which started to be demanding for deployed systems. An increasing number of real-time applications, for instance robotic systems, requires security ensurance for protecting real-time sensive data, against unauthorized access.\cite{lin2009static} 
% Previously, robotic systems were protected using a static environment, usually closed networks. The need for robotics evolution demanded a change of approach, where systems must be acessible and extendable from the public network, allowing flexibility within the network, at the cost of security.

% This lead to the creation of Robot Operating System 2, which continues to provide a simple, uniform message passing interface to allow components to communicate with each other, now implemented using the Data Distribution Service (DDS)\cite{3} specification protocol as its middleware, leveraging for its messaging architecture. Issues concerning system integration and scalability are mitigated by DDS various implementations, due to the several transport configurations provided, making it suitable for real-time distributed systems. DDS also provides a security specification, called DDS-Security, used by multiple implementations, and ROS2 makes use of them to provide features to the deployed robotic systems.\cite{8442103}


\section{Objectives and Contributions}

The first goal of this thesis rests in introducing concepts around the Robot Operating System, contextualizing the evolution behind its framework towards achieving security, where the former version of ROS lacked due to the focus on flexibility. Since ROS2 has been developed over the DDS framework, as its communication middleware, DDS must be properly understood before considering the security aspects. % To do so, simple examples will be previously introduced, in order to provide ROS-based context to the reader. The domain of autonomous systems, where security is of extreme relevance, namely the Autoware \cite{8443742} ROS2 platform for self-driving vehicles will also be introduced. To understand SROS2 we intend to configure and run a realistic case study related to Autoware with security guarantees. 
The DDS Security standard functionality is evaluated, as well as how security is integrated into ROS2. Since security issues, concerning public networks, are recent to the robotics domain, ROS2 security network design should be analyzed structurally.

Security configuration related to SROS2 toolset will be provided in this chapter, supported by an example that accounts multiple security features, those being authentication, encryption and, most importantly, access control, applying restriction constraints to the network and its participants, that by default are not controlled.

The second goal is to extend a previously proposed \cite{9341085} formalization of ROS applications in Alloy \cite{alloy-6, lwspecification} to also take into consideration the security configuration defined with SROS2. Using this extension, we intend to explore the viability of verifying simple information-flow security properties. For instance, to ensure that no commands to the vehicle motor can be sent via the infotainment system.

The final goal is to automate the extraction of such formal Alloy models from the configuration files of a ROS2 application, in order to obtain a prototype tool that can be used by roboticists to easily detect security configuration issues.

\section{Document Structure}

% The current dissertation structure is divided into four different chapters. Chapter (\ref{c:ros}) introduces all the concepts related to Robot Operating System, and its evolution as robotic development framework towards achieving system security. The current provided work regarding ROS and security in ROS is also presented in this chapter. Chapter (\ref{c:alloy}) introduces the Alloy framework, as its specification language supported by a concrete example case. Chapter (\ref{c:currWork}) presents the approach developed during this work, which allows
% the automatic verification of system-wide safety properties, considering security in ROS2 applications. % Chapter (\ref{c:evaluation}) evaluates the latter approach supported by a concrete example. Chapter (\ref{c:conc}) presents the thesis conclusions, alongisde possible future aproaches complementing the current work.

The current dissertation structure is divided into three different chapters. Chapter (\ref{c:ros}) introduces all the concepts related to Robot Operating System, and its evolution as robotic development framework towards achieving system security. Chapter (\ref{c:alloy}) introduces the Alloy framework, as its specification language supported by a concrete example case. Chapter (\ref{c:currWork}) presents previous developed work that covers concepts that are also addressed within this dissertation, as well as, expected considerations about the future work. 

\chapter{Software Development in ROS2}\label{c:ros}

Robotic systems have emerged into several scenarios, where its usage ranges between basic processes automation, up to full performance over critical tasks, consequently causing the complexity increase in these domains. Concerning the complexity behind writting software code, due to the widely variety presented in the robot's hardware as in fields of action \cite{cousins2011exponential}, Robotic Operation System (ROS) presents itself as a middleware system, created to facilitate robotic systems development.

In ROS, software flexibility was valued above all else, meaning that values like security were not duly valued, so ROS-based applications tend to face increased security risks, exposing the whole robotic network. As ROS became a standard for many robotic systems, and due to the scale and scope of the robotics growth, security ensurance must be addressed as a developing priority. \cite{diluoffo2018robot, kim2018security}

The upgraded version of ROS, Robot Operating System 2 (ROS2), presents itself as a framework for developing robotic systems, supported by a standard, the Data Distribution Service (DDS), where multiple middleware implementations are built over this standard, providing ROS applications multiple DDS-based specfications, as well as valued Quality of Service (QoS) settings over the transport configuration. 

The DDS-Security specification \cite{dds-s}, embedded by every DDS implementation supported by ROS, supplies multiple plugins regarding the security domain. As result, ROS2 yields a wider command toolset compared to the former version of ROS, as they bring forth to a toolset, the Secure Robot Operating System 2 (SROS2) toolset, concerning the security functionality that DDS-Security plugins offer.

This chapter introduces necessary background information over the major concepts on which this thesis rests. First, it is presented a detailed introduction to the concepts around Robot Operating System (ROS), as well as the evolution approach that ROS faced towards providing security to its deployed systems. Regarding this goal, Data Distribution Service (DDS) and its integration on Robot Operating System 2 (ROS2) must be contextualized beforehand.


\section{Architecture Considerations}

The Robot Operating System was created by a collaborative open-source community, that has undergone rapid development \cite{cousins2011exponential}, to contribute in the advancement of cyber physical systems, serving as developer enhancer for the world of robotic applications. \cite{diluoffo2018robot, intro-ros}

Fundamentally, ROS is a middleware, as it provides a custom serialization format, a custom transport protocol as well as a custom central discovery mechanism, presenting itself as a distributed layer between the top application layer and the operating system layer. ROS was designed to provide as much as modularity and composability to the application layer \cite{casini2019response} as possible, allowing ROS applications to be built over several software modules, as independent computing processes called \textit{nodes}, that compose together to fulfill the deployment characteristics of the corresponding robot. \cite{maruyama2016exploring} 

\subsection{Former Architecture}

The former communication architecture supported by ROS focused on a centralization perspective, as it had a implementation of a \textit{Master node}, that controlled every aspect of the communication establishment. Every information exchange between nodes had to go through this master, as these nodes must also be able to address the ROS Master's location.

Due to the sheer wide capabilities controlled by the master, this centralization approach fits the purposes of a research tool, as it is simpler to monitor and analyse the system behaviour. This communication architecture, however, does not scale well since it is heavily reliant on the master node's availability, making it unsuitable for safety-critical and real-time applications. If the master fails, the entire system fails, representing a single point of failure and a huge performance bottleneck.

\begin{figure}[H]
  \centering
  \includegraphics[width=0.6\linewidth]{images/former-ros1-architecture.png}
  \caption{Robot Operating System architecture.}
  \label{fig:ros1-architecture}
\end{figure}

Many research communities tried to fix these real-time issues by proposing potential solutions, while supporting the same architecture design. However, these solutions did not fully accomplished the needs of real-time applications. So, it became clear to the ROS community that the framework had architectural limitations that could not be rearranged using the same design approach. \cite{maruyama2016exploring}

Robot Operating System 2 comes as a complete refactoring of ROS, with the aim of increase the framework's real-time capabilities, by allowing the development of time-critical control over ROS, as it moves away from the former architectural design towards the implementation of an external middleware that can support the production needs of the outgrowing robotic systems. \cite{kim2018security, casini2019response}

\subsection{Data Distribution Service}

The Data Distributed System \cite{3}, simply known as DDS, is a Object Management Group (OMG) middleware standard, resulted from the need of better interoperability between different vendors middleware frameworks, directly addressing data communication between nodes that belong to a \textit{publish-subscribe} communication architecture, for real-time and embedded systems. 

A middleware, such as DDS, aims to ease the complexity behind creating each sytem's own middleware, by handling relevant aspects like network configuration, communication establishment, data sharing and low-level details. As a result, system developers can mainly focus on their applications purposes, rather than concerning about information moving across levels. \cite{dds-what-is} 

DDS uses the Data-Centric Publish Subscribe (DCPS) model as its communication model approach. DCPS is based on a \textit{publish-subscribe pattern}, where the \textit{data-centric} messaging technique is implemented. It conceptually creates a virtual \textit{global data space}, acessible by any DDS-based application, where data is properly delivered to the applications which quest for it, saving bandwith and processing power. \cite{3, pardo2005introduction} A domain participant enables an application to participate in the global data space, either as a \textit{publisher} or as a \textit{subscriber}, according to their role on data exchange. \cite{maruyama2016exploring, alaerjan2017modeling, dcps-rtps} 

\begin{figure}[H]
    \centering
    \includegraphics[width=0.6\linewidth]{images/dcps-model.png}
    \caption{DDS architecture: DCPS model with RTPS. Extracted from \cite{maruyama2016exploring}.}
    \label{fig:dcps-model}
\end{figure}

To properly address the data transportation through physical network, DDS offers a wire specification protocol called Real-Time Publish-Subscribe Wire Protocol (RTPS) \cite{rtps}, providing automatic discovery between participants. This protocol also works under a \textit{publish-subscribe} policy over best-effort transports, where data transmission between endpoints is handled. \cite{yun2017data} RTPS allows multiple applications, that could differ on their used DDS implementations, to interoperate with each other as network domain participants. \cite{dcps-rtps, alaerjan2017modeling}

Furthermore, RTPS was designed to make use of \textit{Quality of Service} profiles, where multiple transport policies can be specified that, by default, DDS does not support. This approach offers flexibility over communication configuration and development versatility, allowing the developer to specify whatever QoS satisfies its system's communication needs. \cite{alaerjan2017modeling, diluoffo2018robot, maruyama2016exploring} 

Briefly speaking, DDS leverages the premise of a transport-independent virtualized Data Bus to address network resources' distribution, in which stateful data is distributed through the network. The involved applications can access this data in motion, representing an architecture with no single point of failure, respectively enabling a realiable way of ensuring data integratity. Consequently, by adopting this approach, the load on the network is independent of the number of applications, making it easily scalable.

\begin{figure}[H]
    \centering
    \includegraphics[width=0.6\linewidth]{images/dds-architecture.png}
    \caption{Data Distributed System architecture in a nutshell.}
    \label{fig:dds-architecture-nutshell}
\end{figure}

% Introduzir as implementações e DDS specifications

\subsection{ROS2-DDS Architecture}

As previously stated, the Robot Operating System 2 was developed to address the lack of support for real-time systems that the former ROS provided, mainly due to its architecture design that relied on their own middleware specification. To address this, ROS2 middleware approach is built upon the DDS framework \cite{maruyama2016exploring}, leveraging DDS for its messaging architecture, where communication and transport configuration are handled. 

As far as dependencies are concerned, DDS implementations have light sized dependencies, often related to language implementation libraries, easing the complexity behind installing and running dependencies for ROS developers. \cite{ros-on-dds}

The middleware's on-top layer regards the ROS client library (\textit{rcl}), already implemented in the former ROS architecture. This layer accounts the availability of ROS concepts to the Application layer, as it provides APIs to ease the software implementation by ROS developers. \cite{ros2documentation} As ROS aims to support different programming languages over the same computing context, each language-specficic API must have its corresponding client library (\textit{rclcpp} regarding \textit{C++} and \textit{rclpy} regarding \textit{Python}). The \textit{rcl} accounts these client libraries by abstracting their specification, reducing code duplication.   
\cite{rcl, casini2019response}


\begin{figure}[H]
    \centering
    \includegraphics[width=\linewidth]{images/ros2-architecture.png}
    \caption{ROS2 framework architecture.}
    \label{fig:ros2-architecture}
\end{figure}

Towards supplying a wide range of configurations back to application layer, to vastly cover the robotic applications needs, ROS2 aims to support multiple DDS implementations, in which these implementations API specification might differ from each other (currently, \textit{FastRTPS} by \textit{eProsima}, \textit{Connext} by \textit{RTI}, and \textit{Vortex OpenSplice} by \textit{Adlink}). It should be noted that the DDS implementations are low-level of abstraction, strictly defined by its corresponding vendor's API. DDS only defines fundamental procedures at a higher degree of abstraction.  

In order to abstract \textit{rcl} from the specifications complexity of these implementations APIs, an DDS-agnostic interface is being introduced, the \textit{rmw} (ROS MiddleWare) interface \cite{casini2019response}, allowing portability among DDS vendors, which consequently enables ROS developers to interpolate DDS implementations, based on their applications needs during runtime. The information flow through the middleware layer is done over structure mapping between ROS and DDS data models, addressed by the \textit{rmw}, regarding the DDS implementation that is being considered at runtime.

\subsection{Computation Graph}

From a logical perspective \cite{casini2019response}, ROS applications are composed of many software modules that operate as computation nodes, allowing its participation into the ROS global data space. The primarily use of publish-subscribe model approach as communication type, through \textit{message-passing} patterns, confers additional concept complexity to the application architecture, where the latter can be naturally represented as a \textit{computation graph}. 

The application's computation graph presents itself as a graphical network, where runtime named entities have their unique role when it comes to data distribution. The mainly used network entities are \textit{Node Instances}, \textit{Topics} and \textit{Interfaces}, which will be covered in this presented section.

\subsubsection{Node Instances}

The application development is done over package orchestrating, where each logically represents a useful software module. Packages might be compromised by numerous \textit{nodes}, that can be perceived as processes that will likely perform computation over the network. It is worth mentioning that, nodes can be connected within a single package or between multiple packages, as they are built over their corresponding packages.

The network is comprised by many nodes, running simultaneously and exchanging data between them, where each node addresses its corresponding network module purpose. Fault tolerance features are guaranteed as nodes have their corresponding unique name, allowing communication in an unambiguous manner, which confers a suitable approach when developing a complex robotic system.

The notable usage o callback functions provide great functionality when it comes to manage the node's behaviour in the communication process. Additionally, \textit{timers} can also be used, since they provide a useful way of managing these callbacks, by time-assigning.

\subsubsection{Communication}

Message-passing is the primary means by which nodes communicate with one another. The \textit{message} definition is a well-typed data structure, which commonly characterizes every data structure concerning the information exchange between nodes. A message is defined by its data type, also known as its \textit{interface}, which can either be primitive (\textit{integer}, \textit{string}, \textit{boolean}, among others), or defined by a complex data structure, where multiple data types are assigned to their corresponding variables. 

ROS computation graph provides \textit{3} different ways of establish node communication, those being \textit{Topics}, \textit{Actions} and \textit{Services}, where each one has its different corresponding interface, specified in different folders with unique namespaces. 

\textit{Topics} are perhaps the most common method, naturally perceived as middle-communication buses, over which messages are passed through. As semantic approach, communication through topics is handled by the publishing-subscribing pattern. A node publishes the message to any number of topics, that are then subscribed by nodes that want to get access to that message. Topics provide a multicast routing scheme, where publish data is casted into the multiple nodes that are subscribed to the topic. 

\begin{figure}[H]
    \centering
    \includegraphics[width=0.4\linewidth]{images/ros2-topics.png}
    \caption{ROS2 communication behaviour over \textit{topics}.}
    \label{fig:ros2-topics}
\end{figure}

A specific \textit{topic} is created upon specifying its entity name over either a publisher or a subscriber callback instance. Whenever a node creates a publisher, intentionally instantiated to publish a message through a specified topic, \textit{roscore} is used to advertise the latter, enabling message passing to the corresponding topic subscribers. Message processing is done via the node's callback functions, which are activated upon message receipt, as it can also be utilized for publishing purposes. \cite{casini2019response}

Even though \textit{topics} are the most conventional way of communication, due to its multicast scheme, subscribers can not be identified by the publishers, so logging and synchronization becomes rather difficult.

The use of \textit{services} enables a client node, that can also be seen as a topic subscriber, to request data from a server, that likewise a topic publisher, furnish data through a service. The data is only provided when the client node makes a request. Each service is always linked to just one server node, and does not maintain active connections. To address the service stalling that the former ROS issued due to the service's synchronization nature, ROS2 services are asynchronous, since it is possible to specify a callback function that is triggered when the service server responds back to the client.

Other notable way of exchanging data is by setting goals through \textit{Actions}. Actions, likewise services, also uses a client-server model, but they were design for other purposes rather than only processing a request and sending back a response. Actions are intended to process long-running tasks, where the client sends a goal request to the server node, that confirms the receiving of this goal. Before returning a response back to the client, the server can send feedback back to the client. 

\subsubsection{Launch Files}

A conventional way of deploying a ROS application is through the use of \textit{launch files}, enabling the multi-configuration over entire robotic applications, where network involved nodes can be individually pre-configurated. Therefore, ROS makes use of the \textit{roslaunch} to automatically initialize the whole network, simultaneously launching each node. This provides a simpler way of monitoring the system nodes. In the Figure \ref{fig:ros-lf} is depicted a launch file example regarding an application composed by \textit{4} nodes.

Additional node configuration, such as name remapping and parameter adjustments, can be specified under the \textit{args} tag, which offers great functionality to the launching process. 

Distinctive namespaces allow the system to start the nodes, without any name nor topic name conflicts. However, this technique has some flaws attached, since it does not furnish a way of launching nodes in a separated terminal, often needed for user interaction purposes, like input reading.

\subsubsection{Parameters}   

Another relevant concept behind ROS is the existence of nodes \textit{parameters}, that allows individual configuration of the network nodes. In the former version of ROS, the node parameters were controlled by a global \textit{parameter server}, managed by its corresponding ROS Master. However, in ROS2 each node declares and manages its own parameters, by using the predefined commands \textit{get} and \textit{set}. Additionally, using a parameter function callback, the node's parameters can easily be edited.
        
\subsubsection{Node Composition}  

Usually a node is attached to a single process, but it is possible to combine multiple nodes into a single process, structurally abstracting some network parts, while improving the network's performance. However, there is a slight difference about how ROS and ROS2 approaches the node composition. In the former version of ROS, node composition was done over the combination of \textit{nodelets}, intentionally designed to ease the cost of overusing TCP for message-passing between nodes. Supported by the former idea of \textit{nodelets}, ROS2 introduces the \textit{components} as software code compiled into shared libraries, that can be loaded into a \textit{component container} process at runtime in the network, ensuring node composition. Node composition could also be applied for security matters. Suppose a scenario where multiple nodes respect the same security policies. By combining them into a single process, the mapping into this set of rules would be direct, easing the usage of security enclaves.
           

\section{Security}

The deployment of real-time systems implies critical concerning about safety and security \cite{maruyama2016exploring}, resulting of the demanding time-critical scenarios. Robotic systems fall under the umbrella of this broad system definition, as they feature unique cyber vulnerabilities related to its integration over highly networked environments, that confers great importance on exposing critical time-reliant scenarios. \cite{mcclean2013preliminary, dieber2017security} 

The network security evaluation in a system is done over applying several analyzing techniques. Generally, these techniques do not cover every security aspect, as new vulnerabilities arise from the technology evolution. \cite{kaeo2004designing}
The appliance of security countermeasures techniques upon configuring the system's network confers a critical step when aiming towards achieving security.

Within this vast topic, several avenues of endeavor come to mind, each deserving of a substantial study. Network security entails pre-exploration of the system's network through practical networking security techniques, such as intrusion detection and traffic analysis. \cite{marin2005network} However, due to the high non-linearity and complexity of real-time systems, implementing such a thorough analysis method in near real-time remains a significant difficulty task. \cite{diao2009design}


\subsection{Security Integration in ROS2}

As aforementioned, ROS middleware faces known vulnerabilities due to its architecture model nature. The former ROS communication is built around TCP ports, allowing robots to be built as network distributed modules. As a result, techniques such as port scanning is usually used to expose the data itself. Due to the ROS master role in the communication architecture, and its ability to connect to other nodes, exposing this node poses a critical threat over the whole network. \cite{8794451} 

There was also worries regarding how ROS handled node communication. Network security may be jeopardized, as a result of the publish-subscribe pattern transparency, where node-to-node communications are settled in plain text, making data content vulnerable to unauthorized usage. \cite{kim2018security}

As result of the \textit{Data Distribution Service} (DDS) implementation as a flexible middleware interface in the ROS2 architecture, issues regarding security is no longer mainly ROS-dependent. Thus, when it comes to addressing security over communication, and subsequently data protection enhancment, ROS2 is heavily reliant on how the DDS standard is able to manage security. \cite{kim2018security, 8794451}

The \textit{Object Management Group} (OMG) \cite{3}, the already mentioned organization who is responsible for maintaining the DDS standard, accounts security integration by supplying an in-depth security specification, consequently adding features to the already developed DDS standard. The \textit{DDS-Security} is a specification that serves as a security extension to the DDS protocol, defined by a set of plugins (Authentication, Access Control, Cryptographic, Logging, Data Tagging), combined in a \textit{Service Plugin Interface (SPI)} architecture. \cite{8442103, ros-dds-integration}

\begin{figure}[H]
    \centering
    \includegraphics[width=0.7\linewidth]{images/dds-security-architecture.png}
    \caption{DDS-Security Architecture. Extracted from \cite{dds-s}.}
    \label{fig:dds-security-architecture}
\end{figure}

This specification enables its integration by furnishing a \textit{Security Model} supplied to the DDS standard, whereas the \textit{Service Plugin Interface} architecture is responsible for granting plugin enhancment for compliant DDS implementations. Moreover, depending on the security requirements needed for a particular application, these plugins might be adjusted by the latter's runtime DDS implementation. \cite{dds-s}

Every DDS implementation supported by ROS2 makes use of the DDS-Security specification, enabling security over ROS's application environment. Even though ROS2 is deployed without security mechanisms by default \cite{ros-dds-integration}, ROS2 provides a toolset, the \textit{Secure Robot Operating System 2} (SROS2) toolset, extending ROS2's functionality to make use of the DDS-Security functionality. 

The control over these tools are done by \textit{rcl}, providing security over the Application layer, while DDS is capable of providing security over the communication architecture. \cite{kim2018security} The SROS2 configuration is done over applying a set of security files to each ROS2 participant, considering the assignment approach (strict or permissive) that is being used.

Since this security integrity on ROS2 is consider a recent technology implementation \cite{ros-dds-integration}, the developer must be aware of improper configuration, that can still lead to security problems. However, the variety of capabilities in SROS2 toolset attempts to aid with security configuration across environments. 

\subsection{SROS2 Configuration}

To properly introduce the set of tools that SROS2 provides, it follows an application example that will now account the security features, as to provide authentication and encryption over the network communication, as well as access control policies over the application nodes. 

\subsubsection{The \textit{TurtleSim} Application}

For instance, consider a well-known example called \textit{TurtleSim}, which is a simulator typically used for learning ROS, mainly composed by \textit{two nodes}, that perform together towards moving a turtle. Additional nodes were implemented, in order to add complexity to the current network, as to later support security as a proper example.

For understanding reasons, the reader may want to see how the network architecture is organized. ROS2 provides a GUI tool called \textit{rqt}, that assists developers in manipulating the network elements, in a more user-friendly manner. The \textit{rqt} visualizer, \textit{rqt\_graph}, allows the developer to perform analysis over a graphical visualization of the network computation graph.

\begin{figure}[H]
    \centering
    \includegraphics[width=0.8\linewidth]{images/ts_rqt_graph.png}
    \caption{\textit{TurtleSim}'s network graph presented by \textit{rqt\_graph}.}
    \label{fig:ts-rqt-graph}
\end{figure}

The \textit{multiplexer} node handles commands related to turtle's movement, acting as a topic selector between two different subscribed topics, each of them was respectively associated with a priority value. Based on the priority valued, the \textit{multiplexer} node forwards the commands, related to the selected topic, into the \textit{turtlesim} node, triggering the turtle's movement. 

However, \textit{multiplexer} is not exclusive to the \textit{turtlesim} node, as it is still possible to directly publish commands to the topic that handles the turtle's movement, since security policies are yet to be implemented.


\subsubsection{Configuration}

In technically terms, a \textit{keystore} must be initiated beforehand, to provide a secured environment over the network. SROS2 yields a command that permits its creation. A keystore is a created directory where files regarding security are stored. By generating a keystore directory, it may then be sourced and utilized by \textit{rcl} features towards applying security to the application. The \textit{security} additional keyword-flag enables features regarding security matters, concerning the DDS-security artifacts.
            
\begin{lstlisting}[title={\textit{Keystore} creation using the proper SROS2 command.}]
ros2 security create_keystore demo
\end{lstlisting}

Upon the creation of the \textit{demo} keystore, three respective subdirectories are created, where each has their own role when it comes to security enhancement over the network.

\textbullet\  The \textit{enclaves} directory contains the security tools related to each enclave created. An \textit{enclave} is a group of ROS nodes, controlled by the same set of security rules, defined in its corresponding enclave directory.

\textbullet\  The \textit{public} directory contains material that is permissible as public. A Certificate Authority certificate is stored in this directory, related to the CA \textit{public key}. It is used to validate the identity and permissions of each ROS network node by the CA. 

\textbullet\  The \textit{private} directory contains material that is considered private. A Certificate Authority certificate is stored in this directory, related to the CA \textit{private key}. It is used to modify the network policies, such as access permissions, and to add new participants. Similar to the public directory, the CA key corresponding to its identity and permissions can be stored in their corresponding individual directories.

The following exports need to be sourced to force SROS2 security features, as they concern relevant environment variables. The first sourced variable points to the directory root of the keystore, allowing ROS2 to identify where the security artifacts are kept. The second serves as the security enabler. The last variable sets which security strategy will be used when dealing with security files.
            
\begin{lstlisting}[title={SROS2 environment variables.}]
export ROS_SECURITY_KEYSTORE=/demo
export ROS_SECURITY_ENABLE=true
export ROS_SECURITY_STRATEGY=Enforce
\end{lstlisting}
            
\subsubsection{Understanding Security Enclaves}

Once the keystore has been created, the respective enclaves can be implemented. As mentioned, an enclave is a group of nodes that follow the same security policy. Enclaves usage are specified upon execution time, implying that their security artifacts are actually used by running processes.

Typically, a node is perceived as an abstraction of a DDS \textit{participant}. However, by considering node composition, as a reliable way for matching multiple nodes simultaneously to the same enclave, this node perception as participants can not be taken into account, due to causing non-negligible overhead. There is also not convenient to compose nodes as individual participants, as far as operating system's security is concerned, where permission distribution and memory becomes rather difficult to handle.

To address this, each participant must be matched to a node shared context, instead of being directly related to a specific node. Thereby, the initial given definition of an enclave is not totally correct, since a participant can either be perceived as single node or as multiple node shared context. So, each enclave security artifacts are used by its respective DDS participant. 

As long as security is enabled, the whole network must be properly authenticated. Thus, every node within the network must be authenticated, using an enclave as their identifier. Node composition can not be considered in this network, as it is not intended to share topic privileges. Note that, if an enclave was shared by multiple nodes, each node policy would be considered as common policy within the enclave.

\begin{lstlisting}[title={\textit{TurtleSim} enclave creation.}]
ros2 security create_key demo /turtlesim
ros2 security create_key demo /multiplexer
ros2 security create_key demo /keyboard
ros2 security create_key demo /random
\end{lstlisting}

The keystore creation, alongside with their respective enclaves, only ensures security over the network communication, in which node authentication and data encryption are concerned. With the proper use of port scanning tools, data encryption can be easily verified. Authentication is ensured upon the enclave's creation. However, to properly apply security over the \textit{TurtleSim} application, access control policies must be appropriately covered.

\subsubsection{Access Control}

To accurately achieve topic exclusivity, in which the turtle's movement is uniquely concerned by the \textit{multiplexer} node, \textit{access control} policies must be applied. The remaining nodes should be considered untrustworthy, denying any potential undesired turtle's movement.
            
In order to provide access control, each permission file needs to be modified, accounting the network policies restrictions. This is ensured by adding security permissions to these files, with the mandatory signature of the Certificate Authority. A suitable way of editing the permission file, \textit{permissions.xml} (file that dictates how the enclave manages the permissions within the network) is by creating a policy file, that explicitly specifies the set of permissions of each enclave.

Following the \textit{ConArmor} policy language \cite{white2018procedurally}, the \textit{SROS2 policy file} confers a restrict \textit{XML schema}, where security policies bind profiles to access permissions for network objects, granting privileges back to a certain profile. \textit{Profiles} are implemented under the \textit{enclave} declaration, to duly support the node composition into a single process, enabling the possibility of combining multiple profiles, respectively addressing its corresponding node. Typically, each \textit{enclave} declaration is linked to a corresponding ROS node, naturally perceived as a DDS participant. 

\textit{Objects} are classified over a subsystem type, structurally characterized by permissions tags. Then \textit{object privileges} are controlled over access values, either \textit{allow} or \textit{deny}, attributed to their corresponding permissions tags. For instance, consider the \textit{topics} domain, where a profile can either publish or subscribe to that topic. To properly address the allowance of a profile privilege over a topic, the permission tag (either subscribe or publish) must be followed with the \textit{allow} tag.

The policy design approach works under the \textit{Mandatory Access Control} (MAC), that denies any privilege by default. The only way of allowing access to any object, is by explicitly specifying the subject's privilege access. 

\begin{lstlisting}[title={Setting permissions into each enclave.}]
ros2 security create_permission demo /turtlesim policies.xml
ros2 security create_permission demo /multiplexer policies.xml
ros2 security create_permission demo /keyboard policies.xml
ros2 security create_permission demo /random policies.xml
\end{lstlisting}

As it follows, security is enabled within this network as well as policy control over topic's permissions. The network can be easily configured and automatically launched through the execution of a launch file, where \textit{roslaunch} uses this files to perform overall initialization.

\begin{figure}[H]
\begin{lstlisting}
<launch>
    <node name="turtlesim" pkg="default" exec="turtlesim" output="screen" args="--ros-args --enclave /turtlesim" />
    <node name="keyboard" pkg="default" exec="keyboard" output="screen" args="--ros-args --enclave /keyboard" />
    <node name="random" pkg="random" exec="random" args="--ros-args --enclave /random" />
    <node name="multiplexer" pkg="multiplexer" exec="multiplexer" args="--ros-args --enclave /multiplexer" />
</launch>
\end{lstlisting}
\label{fig:ros-lf}
\caption{\textit{TurtleSim} launch file.}
\end{figure}

The network is now sucessfully running as a secured environment, where nodes within this network can not be perceived from outside, neither their topic list. To prove that access control is properly employed, the user may want to try to enhance the turtle movement directly from the random controller, by forcing the remapping of the \textit{low\_topic} to the \textit{main\_topic}. Thus, by attempting to remap the \textit{low\_topic} topic prevents the node from launching, since the random node is only allowed to publish through the \textit{low\_topic}. 

\begin{lstlisting}[title={Attempting the \textit{low\_topic} remap.}, language=xml]
<node name="random" pkg="turtle_random" exec="random" args="--ros-args --enclave /random -r /low_topic:=/main_topic"/>
\end{lstlisting}

% However, if the user forces the inverted remap, it is possible to control the turtle movement directly from the random controller, since no policie has been disrespecteded. Although, the random controller is still publishing to the \textit{low\_topic}, the \textit{main\_topic} in which the turtle movement is concerned is remapped towards the \textit{low\_topic}. This concerns a problem that is not been duly address by the ROS community. It is unreasonable to expect this flexibility in a secured network, since policies initially settled can be easily compromised.

\chapter{Software Development in ROS2}\label{c:ros}

Robotic systems have emerged into several scenarios, where its usage ranges between basic processes automation, up to full performance over critical tasks, consequently causing the complexity increase in these domains. Concerning the complexity behind writting software code, due to the widely variety presented in the robot's hardware as in fields of action \cite{cousins2011exponential}, Robotic Operation System (ROS) presents itself as a middleware system, created to facilitate robotic systems development.

In ROS, software flexibility was valued above all else, meaning that values like security were not duly valued, so ROS-based applications tend to face increased security risks, exposing the whole robotic network. As ROS became a standard for many robotic systems, and due to the scale and scope of the robotics growth, security ensurance must be addressed as a developing priority. \cite{diluoffo2018robot, kim2018security}

The upgraded version of ROS, Robot Operating System 2 (ROS2), presents itself as a framework for developing robotic systems, supported by a standard, the Data Distribution Service (DDS), where multiple middleware implementations are built over this standard, providing ROS applications multiple DDS-based specfications, as well as valued Quality of Service (QoS) settings over the transport configuration. 

The DDS-Security specification \cite{dds-s}, embedded by every DDS implementation supported by ROS, supplies multiple plugins regarding the security domain. As result, ROS2 yields a wider command toolset compared to the former version of ROS, as they bring forth to a toolset, the Secure Robot Operating System 2 (SROS2) toolset, concerning the security functionality that DDS-Security plugins offer.

This chapter introduces necessary background information over the major concepts on which this thesis rests. First, it is presented a detailed introduction to the concepts around Robot Operating System (ROS), as well as the evolution approach that ROS faced towards providing security to its deployed systems. Regarding this goal, Data Distribution Service (DDS) and its integration on Robot Operating System 2 (ROS2) must be contextualized beforehand.


\section{Architecture Considerations}

The Robot Operating System was created by a collaborative open-source community, that has undergone rapid development \cite{cousins2011exponential}, to contribute in the advancement of cyber physical systems, serving as developer enhancer for the world of robotic applications. \cite{diluoffo2018robot, intro-ros}

Fundamentally, ROS is a middleware, as it provides a custom serialization format, a custom transport protocol as well as a custom central discovery mechanism, presenting itself as a distributed layer between the top application layer and the operating system layer. ROS was designed to provide as much as modularity and composability to the application layer \cite{casini2019response} as possible, allowing ROS applications to be built over several software modules, as independent computing processes called \textit{nodes}, that compose together to fulfill the deployment characteristics of the corresponding robot. \cite{maruyama2016exploring} 

\subsection{Former Architecture}

The former communication architecture supported by ROS focused on a centralization perspective, as it had a implementation of a \textit{Master node}, that controlled every aspect of the communication establishment. Every information exchange between nodes had to go through this master, as these nodes must also be able to address the ROS Master's location.

Due to the sheer wide capabilities controlled by the master, this centralization approach fits the purposes of a research tool, as it is simpler to monitor and analyse the system behaviour. This communication architecture, however, does not scale well since it is heavily reliant on the master node's availability, making it unsuitable for safety-critical and real-time applications. If the master fails, the entire system fails, representing a single point of failure and a huge performance bottleneck.

\begin{figure}[H]
  \centering
  \includegraphics[width=0.6\linewidth]{images/former-ros1-architecture.png}
  \caption{Robot Operating System architecture.}
  \label{fig:ros1-architecture}
\end{figure}

Many research communities tried to fix these real-time issues by proposing potential solutions, while supporting the same architecture design. However, these solutions did not fully accomplished the needs of real-time applications. So, it became clear to the ROS community that the framework had architectural limitations that could not be rearranged using the same design approach. \cite{maruyama2016exploring}

Robot Operating System 2 comes as a complete refactoring of ROS, with the aim of increase the framework's real-time capabilities, by allowing the development of time-critical control over ROS, as it moves away from the former architectural design towards the implementation of an external middleware that can support the production needs of the outgrowing robotic systems. \cite{kim2018security, casini2019response}

\subsection{Data Distribution Service}

The Data Distributed System \cite{3}, simply known as DDS, is a Object Management Group (OMG) middleware standard, resulted from the need of better interoperability between different vendors middleware frameworks, directly addressing data communication between nodes that belong to a \textit{publish-subscribe} communication architecture, for real-time and embedded systems. 

A middleware, such as DDS, aims to ease the complexity behind creating each sytem's own middleware, by handling relevant aspects like network configuration, communication establishment, data sharing and low-level details. As a result, system developers can mainly focus on their applications purposes, rather than concerning about information moving across levels. \cite{dds-what-is} 

DDS uses the Data-Centric Publish Subscribe (DCPS) model as its communication model approach. DCPS is based on a \textit{publish-subscribe pattern}, where the \textit{data-centric} messaging technique is implemented. It conceptually creates a virtual \textit{global data space}, acessible by any DDS-based application, where data is properly delivered to the applications which quest for it, saving bandwith and processing power. \cite{3, pardo2005introduction} A domain participant enables an application to participate in the global data space, either as a \textit{publisher} or as a \textit{subscriber}, according to their role on data exchange. \cite{maruyama2016exploring, alaerjan2017modeling, dcps-rtps} 

\begin{figure}[H]
    \centering
    \includegraphics[width=0.6\linewidth]{images/dcps-model.png}
    \caption{DDS architecture: DCPS model with RTPS. Extracted from \cite{maruyama2016exploring}.}
    \label{fig:dcps-model}
\end{figure}

To properly address the data transportation through physical network, DDS offers a wire specification protocol called Real-Time Publish-Subscribe Wire Protocol (RTPS) \cite{rtps}, providing automatic discovery between participants. This protocol also works under a \textit{publish-subscribe} policy over best-effort transports, where data transmission between endpoints is handled. \cite{yun2017data} RTPS allows multiple applications, that could differ on their used DDS implementations, to interoperate with each other as network domain participants. \cite{dcps-rtps, alaerjan2017modeling}

Furthermore, RTPS was designed to make use of \textit{Quality of Service} profiles, where multiple transport policies can be specified that, by default, DDS does not support. This approach offers flexibility over communication configuration and development versatility, allowing the developer to specify whatever QoS satisfies its system's communication needs. \cite{alaerjan2017modeling, diluoffo2018robot, maruyama2016exploring} 

Briefly speaking, DDS leverages the premise of a transport-independent virtualized Data Bus to address network resources' distribution, in which stateful data is distributed through the network. The involved applications can access this data in motion, representing an architecture with no single point of failure, respectively enabling a realiable way of ensuring data integratity. Consequently, by adopting this approach, the load on the network is independent of the number of applications, making it easily scalable.

\begin{figure}[H]
    \centering
    \includegraphics[width=0.6\linewidth]{images/dds-architecture.png}
    \caption{Data Distributed System architecture in a nutshell.}
    \label{fig:dds-architecture-nutshell}
\end{figure}

% Introduzir as implementações e DDS specifications

\subsection{ROS2-DDS Architecture}

As previously stated, the Robot Operating System 2 was developed to address the lack of support for real-time systems that the former ROS provided, mainly due to its architecture design that relied on their own middleware specification. To address this, ROS2 middleware approach is built upon the DDS framework \cite{maruyama2016exploring}, leveraging DDS for its messaging architecture, where communication and transport configuration are handled. 

As far as dependencies are concerned, DDS implementations have light sized dependencies, often related to language implementation libraries, easing the complexity behind installing and running dependencies for ROS developers. \cite{ros-on-dds}

The middleware's on-top layer regards the ROS client library (\textit{rcl}), already implemented in the former ROS architecture. This layer accounts the availability of ROS concepts to the Application layer, as it provides APIs to ease the software implementation by ROS developers. \cite{ros2documentation} As ROS aims to support different programming languages over the same computing context, each language-specficic API must have its corresponding client library (\textit{rclcpp} regarding \textit{C++} and \textit{rclpy} regarding \textit{Python}). The \textit{rcl} accounts these client libraries by abstracting their specification, reducing code duplication.   
\cite{rcl, casini2019response}


\begin{figure}[H]
    \centering
    \includegraphics[width=\linewidth]{images/ros2-architecture.png}
    \caption{ROS2 framework architecture.}
    \label{fig:ros2-architecture}
\end{figure}

Towards supplying a wide range of configurations back to application layer, to vastly cover the robotic applications needs, ROS2 aims to support multiple DDS implementations, in which these implementations API specification might differ from each other (currently, \textit{FastRTPS} by \textit{eProsima}, \textit{Connext} by \textit{RTI}, and \textit{Vortex OpenSplice} by \textit{Adlink}). It should be noted that the DDS implementations are low-level of abstraction, strictly defined by its corresponding vendor's API. DDS only defines fundamental procedures at a higher degree of abstraction.  

In order to abstract \textit{rcl} from the specifications complexity of these implementations APIs, an DDS-agnostic interface is being introduced, the \textit{rmw} (ROS MiddleWare) interface \cite{casini2019response}, allowing portability among DDS vendors, which consequently enables ROS developers to interpolate DDS implementations, based on their applications needs during runtime. The information flow through the middleware layer is done over structure mapping between ROS and DDS data models, addressed by the \textit{rmw}, regarding the DDS implementation that is being considered at runtime.

\subsection{Computation Graph}

From a logical perspective \cite{casini2019response}, ROS applications are composed of many software modules that operate as computation nodes, allowing its participation into the ROS global data space. The primarily use of publish-subscribe model approach as communication type, through \textit{message-passing} patterns, confers additional concept complexity to the application architecture, where the latter can be naturally represented as a \textit{computation graph}. 

The application's computation graph presents itself as a graphical network, where runtime named entities have their unique role when it comes to data distribution. The mainly used network entities are \textit{Node Instances}, \textit{Topics} and \textit{Interfaces}, which will be covered in this presented section.

\subsubsection{Node Instances}

The application development is done over package orchestrating, where each logically represents a useful software module. Packages might be compromised by numerous \textit{nodes}, that can be perceived as processes that will likely perform computation over the network. It is worth mentioning that, nodes can be connected within a single package or between multiple packages, as they are built over their corresponding packages.

The network is comprised by many nodes, running simultaneously and exchanging data between them, where each node addresses its corresponding network module purpose. Fault tolerance features are guaranteed as nodes have their corresponding unique name, allowing communication in an unambiguous manner, which confers a suitable approach when developing a complex robotic system.

The notable usage o callback functions provide great functionality when it comes to manage the node's behaviour in the communication process. Additionally, \textit{timers} can also be used, since they provide a useful way of managing these callbacks, by time-assigning.

\subsubsection{Communication}

Message-passing is the primary means by which nodes communicate with one another. The \textit{message} definition is a well-typed data structure, which commonly characterizes every data structure concerning the information exchange between nodes. A message is defined by its data type, also known as its \textit{interface}, which can either be primitive (\textit{integer}, \textit{string}, \textit{boolean}, among others), or defined by a complex data structure, where multiple data types are assigned to their corresponding variables. 

ROS computation graph provides \textit{3} different ways of establish node communication, those being \textit{Topics}, \textit{Actions} and \textit{Services}, where each one has its different corresponding interface, specified in different folders with unique namespaces. 

\textit{Topics} are perhaps the most common method, naturally perceived as middle-communication buses, over which messages are passed through. As semantic approach, communication through topics is handled by the publishing-subscribing pattern. A node publishes the message to any number of topics, that are then subscribed by nodes that want to get access to that message. Topics provide a multicast routing scheme, where publish data is casted into the multiple nodes that are subscribed to the topic. 

\begin{figure}[H]
    \centering
    \includegraphics[width=0.4\linewidth]{images/ros2-topics.png}
    \caption{ROS2 communication behaviour over \textit{topics}.}
    \label{fig:ros2-topics}
\end{figure}

A specific \textit{topic} is created upon specifying its entity name over either a publisher or a subscriber callback instance. Whenever a node creates a publisher, intentionally instantiated to publish a message through a specified topic, \textit{roscore} is used to advertise the latter, enabling message passing to the corresponding topic subscribers. Message processing is done via the node's callback functions, which are activated upon message receipt, as it can also be utilized for publishing purposes. \cite{casini2019response}

Even though \textit{topics} are the most conventional way of communication, due to its multicast scheme, subscribers can not be identified by the publishers, so logging and synchronization becomes rather difficult.

The use of \textit{services} enables a client node, that can also be seen as a topic subscriber, to request data from a server, that likewise a topic publisher, furnish data through a service. The data is only provided when the client node makes a request. Each service is always linked to just one server node, and does not maintain active connections. To address the service stalling that the former ROS issued due to the service's synchronization nature, ROS2 services are asynchronous, since it is possible to specify a callback function that is triggered when the service server responds back to the client.

Other notable way of exchanging data is by setting goals through \textit{Actions}. Actions, likewise services, also uses a client-server model, but they were design for other purposes rather than only processing a request and sending back a response. Actions are intended to process long-running tasks, where the client sends a goal request to the server node, that confirms the receiving of this goal. Before returning a response back to the client, the server can send feedback back to the client. 

\subsubsection{Launch Files}

A conventional way of deploying a ROS application is through the use of \textit{launch files}, enabling the multi-configuration over entire robotic applications, where network involved nodes can be individually pre-configurated. Therefore, ROS makes use of the \textit{roslaunch} to automatically initialize the whole network, simultaneously launching each node. This provides a simpler way of monitoring the system nodes. In the Figure \ref{fig:ros-lf} is depicted a launch file example regarding an application composed by \textit{4} nodes.

Additional node configuration, such as name remapping and parameter adjustments, can be specified under the \textit{args} tag, which offers great functionality to the launching process. 

Distinctive namespaces allow the system to start the nodes, without any name nor topic name conflicts. However, this technique has some flaws attached, since it does not furnish a way of launching nodes in a separated terminal, often needed for user interaction purposes, like input reading.

\subsubsection{Parameters}   

Another relevant concept behind ROS is the existence of nodes \textit{parameters}, that allows individual configuration of the network nodes. In the former version of ROS, the node parameters were controlled by a global \textit{parameter server}, managed by its corresponding ROS Master. However, in ROS2 each node declares and manages its own parameters, by using the predefined commands \textit{get} and \textit{set}. Additionally, using a parameter function callback, the node's parameters can easily be edited.
        
\subsubsection{Node Composition}  

Usually a node is attached to a single process, but it is possible to combine multiple nodes into a single process, structurally abstracting some network parts, while improving the network's performance. However, there is a slight difference about how ROS and ROS2 approaches the node composition. In the former version of ROS, node composition was done over the combination of \textit{nodelets}, intentionally designed to ease the cost of overusing TCP for message-passing between nodes. Supported by the former idea of \textit{nodelets}, ROS2 introduces the \textit{components} as software code compiled into shared libraries, that can be loaded into a \textit{component container} process at runtime in the network, ensuring node composition. Node composition could also be applied for security matters. Suppose a scenario where multiple nodes respect the same security policies. By combining them into a single process, the mapping into this set of rules would be direct, easing the usage of security enclaves.
           

\section{Security}

The deployment of real-time systems implies critical concerning about safety and security \cite{maruyama2016exploring}, resulting of the demanding time-critical scenarios. Robotic systems fall under the umbrella of this broad system definition, as they feature unique cyber vulnerabilities related to its integration over highly networked environments, that confers great importance on exposing critical time-reliant scenarios. \cite{mcclean2013preliminary, dieber2017security} 

The network security evaluation in a system is done over applying several analyzing techniques. Generally, these techniques do not cover every security aspect, as new vulnerabilities arise from the technology evolution. \cite{kaeo2004designing}
The appliance of security countermeasures techniques upon configuring the system's network confers a critical step when aiming towards achieving security.

Within this vast topic, several avenues of endeavor come to mind, each deserving of a substantial study. Network security entails pre-exploration of the system's network through practical networking security techniques, such as intrusion detection and traffic analysis. \cite{marin2005network} However, due to the high non-linearity and complexity of real-time systems, implementing such a thorough analysis method in near real-time remains a significant difficulty task. \cite{diao2009design}


\subsection{Security Integration in ROS2}

As aforementioned, ROS middleware faces known vulnerabilities due to its architecture model nature. The former ROS communication is built around TCP ports, allowing robots to be built as network distributed modules. As a result, techniques such as port scanning is usually used to expose the data itself. Due to the ROS master role in the communication architecture, and its ability to connect to other nodes, exposing this node poses a critical threat over the whole network. \cite{8794451} 

There was also worries regarding how ROS handled node communication. Network security may be jeopardized, as a result of the publish-subscribe pattern transparency, where node-to-node communications are settled in plain text, making data content vulnerable to unauthorized usage. \cite{kim2018security}

As result of the \textit{Data Distribution Service} (DDS) implementation as a flexible middleware interface in the ROS2 architecture, issues regarding security is no longer mainly ROS-dependent. Thus, when it comes to addressing security over communication, and subsequently data protection enhancment, ROS2 is heavily reliant on how the DDS standard is able to manage security. \cite{kim2018security, 8794451}

The \textit{Object Management Group} (OMG) \cite{3}, the already mentioned organization who is responsible for maintaining the DDS standard, accounts security integration by supplying an in-depth security specification, consequently adding features to the already developed DDS standard. The \textit{DDS-Security} is a specification that serves as a security extension to the DDS protocol, defined by a set of plugins (Authentication, Access Control, Cryptographic, Logging, Data Tagging), combined in a \textit{Service Plugin Interface (SPI)} architecture. \cite{8442103, ros-dds-integration}

\begin{figure}[H]
    \centering
    \includegraphics[width=0.7\linewidth]{images/dds-security-architecture.png}
    \caption{DDS-Security Architecture. Extracted from \cite{dds-s}.}
    \label{fig:dds-security-architecture}
\end{figure}

This specification enables its integration by furnishing a \textit{Security Model} supplied to the DDS standard, whereas the \textit{Service Plugin Interface} architecture is responsible for granting plugin enhancment for compliant DDS implementations. Moreover, depending on the security requirements needed for a particular application, these plugins might be adjusted by the latter's runtime DDS implementation. \cite{dds-s}

Every DDS implementation supported by ROS2 makes use of the DDS-Security specification, enabling security over ROS's application environment. Even though ROS2 is deployed without security mechanisms by default \cite{ros-dds-integration}, ROS2 provides a toolset, the \textit{Secure Robot Operating System 2} (SROS2) toolset, extending ROS2's functionality to make use of the DDS-Security functionality. 

The control over these tools are done by \textit{rcl}, providing security over the Application layer, while DDS is capable of providing security over the communication architecture. \cite{kim2018security} The SROS2 configuration is done over applying a set of security files to each ROS2 participant, considering the assignment approach (strict or permissive) that is being used.

Since this security integrity on ROS2 is consider a recent technology implementation \cite{ros-dds-integration}, the developer must be aware of improper configuration, that can still lead to security problems. However, the variety of capabilities in SROS2 toolset attempts to aid with security configuration across environments. 

\subsection{SROS2 Configuration}

To properly introduce the set of tools that SROS2 provides, it follows an application example that will now account the security features, as to provide authentication and encryption over the network communication, as well as access control policies over the application nodes. 

\subsubsection{The \textit{TurtleSim} Application}

For instance, consider a well-known example called \textit{TurtleSim}, which is a simulator typically used for learning ROS, mainly composed by \textit{two nodes}, that perform together towards moving a turtle. Additional nodes were implemented, in order to add complexity to the current network, as to later support security as a proper example.

For understanding reasons, the reader may want to see how the network architecture is organized. ROS2 provides a GUI tool called \textit{rqt}, that assists developers in manipulating the network elements, in a more user-friendly manner. The \textit{rqt} visualizer, \textit{rqt\_graph}, allows the developer to perform analysis over a graphical visualization of the network computation graph.

\begin{figure}[H]
    \centering
    \includegraphics[width=0.8\linewidth]{images/ts_rqt_graph.png}
    \caption{\textit{TurtleSim}'s network graph presented by \textit{rqt\_graph}.}
    \label{fig:ts-rqt-graph}
\end{figure}

The \textit{multiplexer} node handles commands related to turtle's movement, acting as a topic selector between two different subscribed topics, each of them was respectively associated with a priority value. Based on the priority valued, the \textit{multiplexer} node forwards the commands, related to the selected topic, into the \textit{turtlesim} node, triggering the turtle's movement. 

However, \textit{multiplexer} is not exclusive to the \textit{turtlesim} node, as it is still possible to directly publish commands to the topic that handles the turtle's movement, since security policies are yet to be implemented.


\subsubsection{Configuration}

In technically terms, a \textit{keystore} must be initiated beforehand, to provide a secured environment over the network. SROS2 yields a command that permits its creation. A keystore is a created directory where files regarding security are stored. By generating a keystore directory, it may then be sourced and utilized by \textit{rcl} features towards applying security to the application. The \textit{security} additional keyword-flag enables features regarding security matters, concerning the DDS-security artifacts.
            
\begin{lstlisting}[title={\textit{Keystore} creation using the proper SROS2 command.}]
ros2 security create_keystore demo
\end{lstlisting}

Upon the creation of the \textit{demo} keystore, three respective subdirectories are created, where each has their own role when it comes to security enhancement over the network.

\textbullet\  The \textit{enclaves} directory contains the security tools related to each enclave created. An \textit{enclave} is a group of ROS nodes, controlled by the same set of security rules, defined in its corresponding enclave directory.

\textbullet\  The \textit{public} directory contains material that is permissible as public. A Certificate Authority certificate is stored in this directory, related to the CA \textit{public key}. It is used to validate the identity and permissions of each ROS network node by the CA. 

\textbullet\  The \textit{private} directory contains material that is considered private. A Certificate Authority certificate is stored in this directory, related to the CA \textit{private key}. It is used to modify the network policies, such as access permissions, and to add new participants. Similar to the public directory, the CA key corresponding to its identity and permissions can be stored in their corresponding individual directories.

The following exports need to be sourced to force SROS2 security features, as they concern relevant environment variables. The first sourced variable points to the directory root of the keystore, allowing ROS2 to identify where the security artifacts are kept. The second serves as the security enabler. The last variable sets which security strategy will be used when dealing with security files.
            
\begin{lstlisting}[title={SROS2 environment variables.}]
export ROS_SECURITY_KEYSTORE=/demo
export ROS_SECURITY_ENABLE=true
export ROS_SECURITY_STRATEGY=Enforce
\end{lstlisting}
            
\subsubsection{Understanding Security Enclaves}

Once the keystore has been created, the respective enclaves can be implemented. As mentioned, an enclave is a group of nodes that follow the same security policy. Enclaves usage are specified upon execution time, implying that their security artifacts are actually used by running processes.

Typically, a node is perceived as an abstraction of a DDS \textit{participant}. However, by considering node composition, as a reliable way for matching multiple nodes simultaneously to the same enclave, this node perception as participants can not be taken into account, due to causing non-negligible overhead. There is also not convenient to compose nodes as individual participants, as far as operating system's security is concerned, where permission distribution and memory becomes rather difficult to handle.

To address this, each participant must be matched to a node shared context, instead of being directly related to a specific node. Thereby, the initial given definition of an enclave is not totally correct, since a participant can either be perceived as single node or as multiple node shared context. So, each enclave security artifacts are used by its respective DDS participant. 

As long as security is enabled, the whole network must be properly authenticated. Thus, every node within the network must be authenticated, using an enclave as their identifier. Node composition can not be considered in this network, as it is not intended to share topic privileges. Note that, if an enclave was shared by multiple nodes, each node policy would be considered as common policy within the enclave.

\begin{lstlisting}[title={\textit{TurtleSim} enclave creation.}]
ros2 security create_key demo /turtlesim
ros2 security create_key demo /multiplexer
ros2 security create_key demo /keyboard
ros2 security create_key demo /random
\end{lstlisting}

The keystore creation, alongside with their respective enclaves, only ensures security over the network communication, in which node authentication and data encryption are concerned. With the proper use of port scanning tools, data encryption can be easily verified. Authentication is ensured upon the enclave's creation. However, to properly apply security over the \textit{TurtleSim} application, access control policies must be appropriately covered.

\subsubsection{Access Control}

To accurately achieve topic exclusivity, in which the turtle's movement is uniquely concerned by the \textit{multiplexer} node, \textit{access control} policies must be applied. The remaining nodes should be considered untrustworthy, denying any potential undesired turtle's movement.
            
In order to provide access control, each permission file needs to be modified, accounting the network policies restrictions. This is ensured by adding security permissions to these files, with the mandatory signature of the Certificate Authority. A suitable way of editing the permission file, \textit{permissions.xml} (file that dictates how the enclave manages the permissions within the network) is by creating a policy file, that explicitly specifies the set of permissions of each enclave.

Following the \textit{ConArmor} policy language \cite{white2018procedurally}, the \textit{SROS2 policy file} confers a restrict \textit{XML schema}, where security policies bind profiles to access permissions for network objects, granting privileges back to a certain profile. \textit{Profiles} are implemented under the \textit{enclave} declaration, to duly support the node composition into a single process, enabling the possibility of combining multiple profiles, respectively addressing its corresponding node. Typically, each \textit{enclave} declaration is linked to a corresponding ROS node, naturally perceived as a DDS participant. 

\textit{Objects} are classified over a subsystem type, structurally characterized by permissions tags. Then \textit{object privileges} are controlled over access values, either \textit{allow} or \textit{deny}, attributed to their corresponding permissions tags. For instance, consider the \textit{topics} domain, where a profile can either publish or subscribe to that topic. To properly address the allowance of a profile privilege over a topic, the permission tag (either subscribe or publish) must be followed with the \textit{allow} tag.

The policy design approach works under the \textit{Mandatory Access Control} (MAC), that denies any privilege by default. The only way of allowing access to any object, is by explicitly specifying the subject's privilege access. 

\begin{lstlisting}[title={Setting permissions into each enclave.}]
ros2 security create_permission demo /turtlesim policies.xml
ros2 security create_permission demo /multiplexer policies.xml
ros2 security create_permission demo /keyboard policies.xml
ros2 security create_permission demo /random policies.xml
\end{lstlisting}

As it follows, security is enabled within this network as well as policy control over topic's permissions. The network can be easily configured and automatically launched through the execution of a launch file, where \textit{roslaunch} uses this files to perform overall initialization.

\begin{figure}[H]
\begin{lstlisting}
<launch>
    <node name="turtlesim" pkg="default" exec="turtlesim" output="screen" args="--ros-args --enclave /turtlesim" />
    <node name="keyboard" pkg="default" exec="keyboard" output="screen" args="--ros-args --enclave /keyboard" />
    <node name="random" pkg="random" exec="random" args="--ros-args --enclave /random" />
    <node name="multiplexer" pkg="multiplexer" exec="multiplexer" args="--ros-args --enclave /multiplexer" />
</launch>
\end{lstlisting}
\label{fig:ros-lf}
\caption{\textit{TurtleSim} launch file.}
\end{figure}

The network is now sucessfully running as a secured environment, where nodes within this network can not be perceived from outside, neither their topic list. To prove that access control is properly employed, the user may want to try to enhance the turtle movement directly from the random controller, by forcing the remapping of the \textit{low\_topic} to the \textit{main\_topic}. Thus, by attempting to remap the \textit{low\_topic} topic prevents the node from launching, since the random node is only allowed to publish through the \textit{low\_topic}. 

\begin{lstlisting}[title={Attempting the \textit{low\_topic} remap.}, language=xml]
<node name="random" pkg="turtle_random" exec="random" args="--ros-args --enclave /random -r /low_topic:=/main_topic"/>
\end{lstlisting}

% However, if the user forces the inverted remap, it is possible to control the turtle movement directly from the random controller, since no policie has been disrespecteded. Although, the random controller is still publishing to the \textit{low\_topic}, the \textit{main\_topic} in which the turtle movement is concerned is remapped towards the \textit{low\_topic}. This concerns a problem that is not been duly address by the ROS community. It is unreasonable to expect this flexibility in a secured network, since policies initially settled can be easily compromised.

\chapter{Verification of Information-Flow Properties concerning Security}\label{c:currWork}


\input{conc.tex}

%\section{Citations}
%Example of a citation: \cite{GRM97}, cf.\ this entry in the \Bibtex\ file.
%Another way of citing is \citep{KeR88}
%
%\chapter{The problem and its challenges}
%         The problem and its challenges.
%
%\section{Images}
%	Example of inserting an image as displyed text,
%\begin{center}
%	\includegraphics[width=0.2\textwidth]{img/mei-logo-03.jpg}
%\end{center}
%
%\begin{wrapfigure}{r}{0.25\textwidth}
%	\includegraphics[width=0.2\textwidth]{img/mei-logo-03.jpg}
%\end{wrapfigure}
%\noindent --- wrapped into the text,
%bla-bla bla-bla bla-bla bla-bla bla-bla bla-bla bla-bla bla-bla bla-bla bla-bla
%bla-bla bla-bla bla-bla bla-bla bla-bla bla-bla bla-bla bla-bla bla-bla bla-bla
%bla-bla bla-bla bla-bla bla-bla bla-bla bla-bla bla-bla bla-bla bla-bla bla-bla
%bla-bla bla-bla bla-bla bla-bla bla-bla bla-bla bla-bla bla-bla bla-bla bla-bla
%bla-bla bla-bla bla-bla bla-bla bla-bla bla-bla bla-bla bla-bla bla-bla bla-bla bla-bla bla-bla bla-bla bla-bla
%bla-bla bla-bla bla-bla bla-bla bla-bla bla-bla bla-bla bla-bla bla-bla bla-bla bla-bla bla-bla bla-bla bla-bla
%
%\noindent --- or as a floating body.
%\begin{figure}
%\begin{center}
%	\includegraphics[width=0.5\textwidth]{img/mei-logo-03.jpg}
%\end{center}
%\caption{Caption}
%\end{figure}
%
%You can also use an image as an icon, eg.\ \MEI, in the main tex.
%Click on it to visit the website. It is also listed in the list of terms.
%Another example of an item to appear in the term index: \UM (needs \Makeindex)
%
%\begin{table}[]
%\begin{tabular}{lllll}
% &  &  &  &  \\
% &  &  &  &  \\
% &  &  &  &  \\
% &  &  &  &
%\end{tabular}
%\caption{Caption}
%\end{table}
%
%\part{Core of the dissertation}
%
%\chapter{Contribution}
%	Main result(s) and their scientific evidence
%\section{Introduction}
%\section{Summary}
%
%\chapter{Applications}
%	Application of main result (examples and case studies)
%\section{Introduction}
%\section{Summary}
%
%\chapter{Conclusions and future work}
%	Conclusions and future work.
%\section{Conclusions}
%\section{Prospect for future work}

\bookmarksetup{startatroot} % Ends last part.
\addtocontents{toc}{\bigskip} % Making the table of contents look good.
\cleardoublepage

%----------------- Bibliography (needs bibtex) --------------------------------%
\bibliography{dissertation}
%----------------- Index of terms (needs  makeindex) --------------------------%
\printindex
%------------------------------------------------------------------------------%

	\appendix
	\renewcommand\chaptername{Appendix}

	% Add appendix chapters

%\part{Appendices}
%
%\chapter{Support work}
%	Auxiliary results which are not main-stream
%
%\chapter{Details of results}
%         Details of results whose length would compromise readability of main text.
%
%\chapter{Listings}
%	Should this be the case

%\chapter{Tooling}
%	(Should this be the case)

%	Anyone using \Latex\ should consider having a look at \TUG,
%	the \tug{\TeX\ Users Group}

\begin{backcover}
\thispagestyle{empty} \pagecolor{white} \textcolor{black} {\fontfamily{phv}\fontseries{mc}\selectfont ~\vfill
\noindent
NB: place here information about funding, FCT project, etc in which the work is framed. Leave empty otherwise.
%
\vfill ~}
\end{backcover}

\end{document}

