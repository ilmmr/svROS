\chapter{Future Work}\label{c:currWork}

This chapter provides a brief introduction over the problems that this dissertation aims to tackle, supported by a graphical dissertation schedule, in which the main development tasks are organized over time.

Following the \textit{Literature review} period where background concepts were duly studied and the pre-dissertation was written, it is expected that within the next first month security-related discussions will take place as major priority, due to its critical role on what this dissertation aims to tackle. Properties regarding security will be properly supported by ROS examples. Thus, it is intended to find concrete example that fails to address a specific security property, where the latter would be previously evaluated in a speculative manner. 

Afterwards, the already discussed schedule and respective assigned timings can proceed its predefined order. Here, ROS2 applications architectures will be formalized in Alloy, where the above's considerations about security and SROS2 configuration specification will be taken into account upon the modelling process. Then, it is expected to propose  a technique to specify and verify information-flow security properties on top of the proposed Alloy formalization.

Both \textit{Evaluation} and \textit{Implementation} periods will follow the latter's objectives. The whole timetable is displayed bellow in the form of a table, with the times evenly split among the several periods. 